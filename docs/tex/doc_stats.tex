\section{\texorpdfstring{\texttt{fitr.stats}}{fitr.stats}}\label{fitr.stats}

Functions for statistical analyses.

\subsection{bic}\label{bic}

\begin{Shaded}
\begin{Highlighting}[]
\NormalTok{fitr.stats.model_evaluation.bic(log_prob, nparams, ntrials)}
\end{Highlighting}
\end{Shaded}

Bayesian Information Criterion (BIC)

Arguments:

\begin{itemize}
\tightlist
\item
  \textbf{log\_prob}: Log probability
\item
  \textbf{nparams}: Number of parameters in the model
\item
  \textbf{ntrials}: Number of trials in the time series
\end{itemize}

Returns:

Scalar estimate of BIC.

\begin{center}\rule{0.5\linewidth}{\linethickness}\end{center}

\subsection{lme}\label{lme}

\begin{Shaded}
\begin{Highlighting}[]
\NormalTok{fitr.stats.model_evaluation.lme(log_prob, nparams, hess_inv)}
\end{Highlighting}
\end{Shaded}

Laplace approximation to the log model evidence

Arguments:

\begin{itemize}
\tightlist
\item
  \textbf{log\_prob}: Log probability
\item
  \textbf{nparams}: Number of parameters in the model
\item
  \textbf{hess\_inv}: Hessian at the optimum (shape is \(K \times K\))
\end{itemize}

Returns:

Scalar approximation of the log model evidence

\begin{center}\rule{0.5\linewidth}{\linethickness}\end{center}

\subsection{pearson\_rho}\label{pearson_rho}

\begin{Shaded}
\begin{Highlighting}[]
\NormalTok{fitr.stats.correlations.pearson_rho(X, Y, comparison}\OperatorTok{=}\StringTok{'diagonal'}\NormalTok{)}
\end{Highlighting}
\end{Shaded}

Linear (Pearson) correlation coefficient.

Will compute the following formula

\[
\rho = \frac{\mathbf x^\top \mathbf y}{\lVert \mathbf x 
Vert \cdot \lVert \mathbf y 
Vert}
\]

where each vector \(\mathbf x\) and \(\mathbf y\) are rows of the
matrices \(\mathbf X\) and \(\mathbf Y\), respectively.

Also returns a two-tailed p-value where the hypotheses being tested are

\[
H_o: \rho = 0
\]

\[
H_a: \rho \neq 0
\]

and where the test statistic is

\[
T = \frac{\rho \sqrt{n_s-2}}{\sqrt{1 - \rho^2}}
\]

and the p-value is thus

\[
p = 2*(1 - \mathcal T(T, n_s-2))
\]

given the CDF of the Student T-distribution with degrees of freedom
\(n_s-2\).

Arguments:

\begin{itemize}
\tightlist
\item
  \textbf{X}: \texttt{ndarray((nsamples,\ nfeatures))} of dimension 1 or
  2. If \texttt{X} is a 1D array, it will be converted to 2D prior to
  computation
\item
  \textbf{Y}: \texttt{ndarray((nsamples,\ nfeatures))} of dimension 1 or
  2. If \texttt{Y} is a 1D array, it will be converted to 2D prior to
  computation
\item
  \textbf{comparison}: \texttt{str}. Here
  \texttt{\textquotesingle{}diagonal\textquotesingle{}} computes
  correlations individually, column-for-column between matrices.
  Otherwise \texttt{\textquotesingle{}pairwise\textquotesingle{}}
  computes pairwise correlations between columns in \texttt{X} and
  \texttt{Y}.
\end{itemize}

Returns:

\begin{itemize}
\tightlist
\item
  \textbf{rho}: \texttt{ndarray((nfeatures,))}. Correlation
  coefficient(s). Will be an \texttt{X.shape{[}1{]}} by
  \texttt{Y.shape{[}1{]}} matrix if
  \texttt{comparison=\textquotesingle{}pairwise\textquotesingle{}}
\item
  \textbf{p}: \texttt{ndarray((nfeatures,))}. P-values for correlation
  coefficient(s). Will be an \texttt{X.shape{[}1{]}} by
  \texttt{Y.shape{[}1{]}} matrix if
  \texttt{comparison=\textquotesingle{}pairwise\textquotesingle{}}
\end{itemize}

TODO:

\begin{itemize}
\tightlist
\item
  {[} {]} Create error raised when X and Y are not same dimension
\end{itemize}

\begin{center}\rule{0.5\linewidth}{\linethickness}\end{center}

\subsection{spearman\_rho}\label{spearman_rho}

\begin{Shaded}
\begin{Highlighting}[]
\NormalTok{fitr.stats.correlations.spearman_rho(X, Y, comparison}\OperatorTok{=}\StringTok{'diagonal'}\NormalTok{)}
\end{Highlighting}
\end{Shaded}

Spearman's rank correlation

Note this function takes correlations between the columns of \texttt{X}
and \texttt{Y}.

Arguments:

\begin{itemize}
\tightlist
\item
  \textbf{X}: \texttt{ndarray((nsamples,\ nfeatures))} of dimension 1 or
  2. If \texttt{X} is a 1D array, it will be converted to 2D prior to
  computation
\item
  \textbf{Y}: \texttt{ndarray((nsamples,\ nfeatures))} of dimension 1 or
  2. If \texttt{Y} is a 1D array, it will be converted to 2D prior to
  computation
\item
  \textbf{comparison}: \texttt{str}. Here
  \texttt{\textquotesingle{}diagonal\textquotesingle{}} computes
  correlations individually, column-for-column between matrices.
  Otherwise \texttt{\textquotesingle{}pairwise\textquotesingle{}}
  computes pairwise correlations between columns in \texttt{X} and
  \texttt{Y}.
\end{itemize}

Returns:

\begin{itemize}
\tightlist
\item
  \textbf{rho}: \texttt{ndarray((nfeatures,))}. Correlation
  coefficient(s). Will be an \texttt{X.shape{[}1{]}} by
  \texttt{Y.shape{[}1{]}} matrix if
  \texttt{comparison=\textquotesingle{}pairwise\textquotesingle{}}
\item
  \textbf{p}: \texttt{ndarray((nfeatures,))}. P-values for correlation
  coefficient(s). Will be an \texttt{X.shape{[}1{]}} by
  \texttt{Y.shape{[}1{]}} matrix if
  \texttt{comparison=\textquotesingle{}pairwise\textquotesingle{}}
\end{itemize}

\begin{center}\rule{0.5\linewidth}{\linethickness}\end{center}

\subsection{linear\_regression}\label{linear_regression}

\begin{Shaded}
\begin{Highlighting}[]
\NormalTok{fitr.stats.linear_regression.linear_regression(X, y, add_intercept}\OperatorTok{=}\VariableTok{True}\NormalTok{, scale_x}\OperatorTok{=}\VariableTok{False}\NormalTok{, scale_y}\OperatorTok{=}\VariableTok{False}\NormalTok{)}
\end{Highlighting}
\end{Shaded}

Performs ordinary least squares linear regression, returning MLEs of the
coefficients

\subsection{Hypothesis testing on the
model}\label{hypothesis-testing-on-the-model}

Compute sum of squares:

\[
SS_R  = (\mathbf y - \bar{y})^      op (\mathbf y - \bar{y})
\]

\[
SS_{Res} = \mathbf y^\top \mathbf y - \mathbf w^\top \mathbf X^\top \mathbf y
\]

\[
SS_T = \mathbf y^\top \mathbf y - \frac{(\mathbf 1^\top \mathbf y)^\top}{n_s}
\]

The test statistic is defined as follows:

\[
F = \frac{SS_R (n-k-1)}{SS_{Res} k} \sim F(k, n-k-1)
\]

The adjusted \(R^2\) is

\[
R^2_{Adj} = 1 - \frac{SS_R (n-1)}{SS_T (n-k-1)}
\]

\subsection{Hypothesis testing on the
coefficients}\label{hypothesis-testing-on-the-coefficients}

The test statistic is

\[
\frac{w_i}{SE(w_i)} \sim StudentT(n-k-1)
\]

Arguments:

\begin{itemize}
\tightlist
\item
  \textbf{X}: \texttt{ndarray((nsamples,\ nfeatures))}. Predictors
\item
  \textbf{y}: \texttt{ndarray(nsamples)}. Target
\item
  \textbf{add\_intercept}: \texttt{bool}. Whether to add an intercept
  term (pads on LHS of \texttt{X} with column of ones)
\item
  \textbf{scale\_x}: \texttt{bool}. Whether to scale the columns of
  \texttt{X}
\item
  \textbf{scale\_y}: \texttt{bool}. Whether to scale the columns of
  \texttt{y}
\end{itemize}

Returns:

\texttt{LinearRegressionResult}

\begin{center}\rule{0.5\linewidth}{\linethickness}\end{center}

\subsection{kruskal\_wallis}\label{kruskal_wallis}

\begin{Shaded}
\begin{Highlighting}[]
\NormalTok{fitr.stats.nonparametric.kruskal_wallis(x, g, dist}\OperatorTok{=}\StringTok{'beta'}\NormalTok{)}
\end{Highlighting}
\end{Shaded}

Kruskal-Wallis one-way analysis of variance (one-way ANOVA on ranks)

Arguments:

\begin{itemize}
\tightlist
\item
  \textbf{x}: \texttt{ndarray(nsamples)}. Vector of data to be compared
\item
  \textbf{g}: \texttt{ndarray(nsamples)}. Group ID's
\item
  \textbf{dist}:
  \texttt{str\ \{\textquotesingle{}chi2\textquotesingle{},\ \textquotesingle{}beta\textquotesingle{}\}}.
  Which distributional approximation to make
\end{itemize}

Returns:

\begin{itemize}
\tightlist
\item
  \textbf{T}: \texttt{float}. Test statistic
\item
  \textbf{p}: \texttt{float}. P-value for the comparison
\end{itemize}

\begin{center}\rule{0.5\linewidth}{\linethickness}\end{center}

\subsection{conover}\label{conover}

\begin{Shaded}
\begin{Highlighting}[]
\NormalTok{fitr.stats.nonparametric.conover(x, g, alpha}\OperatorTok{=}\FloatTok{0.05}\NormalTok{, adjust}\OperatorTok{=}\StringTok{'bonferroni'}\NormalTok{)}
\end{Highlighting}
\end{Shaded}

Conover's nonparametric test of homogeneity.

Arguments:

\begin{itemize}
\tightlist
\item
  \textbf{x}: \texttt{ndarray(nsamples)}. Vector of data to be compared
\item
  \textbf{g}: \texttt{ndarray(nsamples)}. Group ID's
\item
  \textbf{alpha}: \texttt{0\ \textless{}\ float\ \textless{}\ 1}.
  Significance threshold
\item
  \textbf{adjust}: \texttt{str}. Method to adjust p-values (see below)
\end{itemize}

Returns:

\begin{itemize}
\tightlist
\item
  \textbf{T}: \texttt{float}. Test statistic
\item
  \textbf{p}: \texttt{float}. P-value for the comparison
\end{itemize}

Notes:

Adjustment methods include the following:

\begin{itemize}
\tightlist
\item
  \texttt{bonferroni} : one-step correction
\item
  \texttt{sidak} : one-step correction
\item
  \texttt{holm-sidak} : step down method using Sidak adjustments
\item
  \texttt{holm} : step-down method using Bonferroni adjustments
\item
  \texttt{simes-hochberg} : step-up method (independent)
\item
  \texttt{hommel} : closed method based on Simes tests (non-negative)
\item
  \texttt{fdr\_bh} : Benjamini/Hochberg (non-negative)
\item
  \texttt{fdr\_by} : Benjamini/Yekutieli (negative)
\item
  \texttt{fdr\_tsbh} : two stage fdr correction (non-negative)
\item
  \texttt{fdr\_tsbky} : two stage fdr correction (non-negative)
\end{itemize}

References:

W. J. Conover and R. L. Iman (1979), On multiple-comparisons procedures,
Tech. Rep.~LA-7677-MS, Los Alamos Scientific Laboratory.

\begin{center}\rule{0.5\linewidth}{\linethickness}\end{center}

\subsection{confusion\_matrix}\label{confusion_matrix}

\begin{Shaded}
\begin{Highlighting}[]
\NormalTok{fitr.stats.confusion_matrix.confusion_matrix(ytrue, ypred)}
\end{Highlighting}
\end{Shaded}

Creates a confusion matrix from some ground truth labels and predictions

Arguments:

\begin{itemize}
\tightlist
\item
  \textbf{ytrue}: \texttt{ndarray(nsamples)}. Ground truth labels
\item
  \textbf{ypred}: \texttt{ndarray(nsamples)}. Predicted labels
\end{itemize}

Returns:

\begin{itemize}
\tightlist
\item
  \textbf{C}: \texttt{ndarray((nlabels,\ nlabels))}. Confusion matrix
\end{itemize}

Example:

In the binary classification case, we may have the following:

\begin{Shaded}
\begin{Highlighting}[]
\ImportTok{from}\NormalTok{ fitr.stats }\ImportTok{import}\NormalTok{ confusion_matrix}

\NormalTok{ytrue }\OperatorTok{=}\NormalTok{ np.array([}\DecValTok{0}\NormalTok{, }\DecValTok{1}\NormalTok{, }\DecValTok{0}\NormalTok{, }\DecValTok{1}\NormalTok{, }\DecValTok{0}\NormalTok{])}
\NormalTok{ypred }\OperatorTok{=}\NormalTok{ np.array([}\DecValTok{1}\NormalTok{, }\DecValTok{1}\NormalTok{, }\DecValTok{0}\NormalTok{, }\DecValTok{1}\NormalTok{, }\DecValTok{0}\NormalTok{])}
\NormalTok{C }\OperatorTok{=}\NormalTok{ confusion_matrix(ytrue, ypred)}
\NormalTok{tn, fp, fn, tp }\OperatorTok{=}\NormalTok{ C.flatten()}
\end{Highlighting}
\end{Shaded}

\begin{center}\rule{0.5\linewidth}{\linethickness}\end{center}
