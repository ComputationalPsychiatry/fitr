\section{\texorpdfstring{\texttt{fitr.hclr}}{fitr.hclr}}\label{fitr.hclr}

Hierarchical convolutional logistic regression (HCLR): A general
analysis method for trial-by-trial behavioural data with covariates.

\subsection{HCLR}\label{hclr}

\begin{Shaded}
\begin{Highlighting}[]
\NormalTok{fitr.hclr.HCLR()}
\end{Highlighting}
\end{Shaded}

Hierarchical Convolutional Logistic Regression (HCLR) for general
behavioural data.

Attributes:

\begin{itemize}
\tightlist
\item
  \textbf{X}: \texttt{ndarray((nsubjects,ntrials,nfeatures))}. The
  ``experience'' tensor.
\item
  \textbf{y}: \texttt{ndarray((nsubjects,ntrials,ntargets))}. Tensor of
  ``choices'' we are trying to predict.
\item
  \textbf{Z}: \texttt{ndarray((nsubjects,ncovariates))}. Covariates of
  interest
\item
  \textbf{V}: \texttt{ndarray((naxes,nfeatures))}. Vectors identifying
  features of interest (i.e.~to compute indices). If
  \texttt{add\_intercept=True}, then the dimensionality of \texttt{V}
  should be \texttt{ndarray((naxes,\ nfeatures+1))}, where the first
  column represents the basis coordinate for the bias.
\item
  \textbf{filter\_size}: \texttt{int}. Number of steps prior to target
  included as features.
\item
  \textbf{loading\_matrix\_scale}: \texttt{float\ \textgreater{}\ 0}.
  Scale of the loading matrix \(\boldsymbol\Phi\), which is assumed that
  \(\phi_{ij} \sim \mathcal N(0, 1)\), with the default scale being 1.
\item
  \textbf{add\_intercept}: `bool'. Whether to add intercept
\item
  \textbf{group\_mean}: \texttt{ndarray}. Samples of the posterior
  group-level mean. \texttt{None} until model is fit
\item
  \textbf{group\_scale}: \texttt{ndarray}. Samples of the posterior
  group-level scale. \texttt{None} until model is fit
\item
  \textbf{loading\_matrix}: \texttt{ndarray}. Samples of the posterior
  loading matrix. \texttt{None} until model is fit
\item
  \textbf{subject\_parameters}: \texttt{ndarray}. Samples of the
  posterior subject-level parameters. \texttt{None} until model is fit
\item
  \textbf{group\_indices}: \texttt{ndarray}. Samples of the posterior
  group-level projections on to the basis. \texttt{None} until model is
  fit
\item
  \textbf{covariate\_effects}: \texttt{ndarray}. Samples of the
  posterior projection of the loading matrix onto the basis.
  \texttt{None} until model is fit
\end{itemize}

\subsection{Notes}\label{notes}

\begin{itemize}
\tightlist
\item
  When presenting \texttt{X} and \texttt{y}, note that the indices of
  \texttt{y} should correspond exactly to the trial indices in
  \texttt{X}, even though the HCLR analysis is predicting a trial ahead.
  In other words, there should be no lag in the \texttt{X}, \texttt{y}
  inputs. The HCLR setup will automatically set up the lag depending on
  how you set the \texttt{filter\_size}.
\end{itemize}

\begin{center}\rule{0.5\linewidth}{\linethickness}\end{center}

\subsubsection{HCLR.fit}\label{hclr.fit}

\begin{Shaded}
\begin{Highlighting}[]
\NormalTok{fitr.hclr.fit(}\VariableTok{self}\NormalTok{, nchains}\OperatorTok{=}\DecValTok{4}\NormalTok{, niter}\OperatorTok{=}\DecValTok{1000}\NormalTok{, warmup}\OperatorTok{=}\VariableTok{None}\NormalTok{, thin}\OperatorTok{=}\DecValTok{1}\NormalTok{, seed}\OperatorTok{=}\VariableTok{None}\NormalTok{, verbose}\OperatorTok{=}\VariableTok{False}\NormalTok{, algorithm}\OperatorTok{=}\StringTok{'NUTS'}\NormalTok{, n_jobs}\OperatorTok{=-}\DecValTok{1}\NormalTok{)}
\end{Highlighting}
\end{Shaded}

Fits the HCLR model

Arguments:

\begin{itemize}
\tightlist
\item
  \textbf{nchains}: \texttt{int}. Number of chains for the MCMC run.
\item
  \textbf{niter}: \texttt{int}. Number of iterations over which to run
  MCMC.
\item
  \textbf{warmup}: \texttt{int}. Number of warmup iterations
\item
  \textbf{thin}: \texttt{int}. Periodicity of sample recording
\item
  \textbf{seed}: \texttt{int}. Seed for pseudorandom number generator
\item
  \textbf{algorithm}:
  \texttt{\{\textquotesingle{}NUTS\textquotesingle{},\textquotesingle{}HMC\textquotesingle{}\}}
\item
  \textbf{n\_jobs}: \texttt{int}. Number of cores to use (default=-1, as
  many as possible and required)
\end{itemize}

\begin{center}\rule{0.5\linewidth}{\linethickness}\end{center}
